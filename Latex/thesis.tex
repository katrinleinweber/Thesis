%----------------------------------------------------------------------------------------
%	PACKAGES EN DOCUMENT CONFIGURATIE
%----------------------------------------------------------------------------------------

\documentclass[a4paper,12pt]{article}
\usepackage{graphicx}
\usepackage[english]{babel}
\usepackage{fancyhdr}
\usepackage{lastpage}
\usepackage{xifthen}
\usepackage{algorithm2e}
\usepackage{lipsum}
\usepackage[hidelinks]{hyperref}
\usepackage[utf8]{inputenc}
\usepackage{amsmath}
\usepackage{amssymb}
\usepackage{float}
\usepackage{multirow}
%----------------------------------------------------------------------------------------
%	HEADER & FOOTER
%----------------------------------------------------------------------------------------
\pagestyle{empty}
  \lhead{\includegraphics[width=4.5cm]{logoUU}}		%Zorg dat het logo in dezelfde map staat
  \rhead{\footnotesize \textsc {\\ \opdracht}}
  \lfoot
    {
	\footnotesize \studentA
	\ifthenelse{\isundefined{\studentB}}{}{\\ \studentB}
	\ifthenelse{\isundefined{\studentC}}{}{\\ \studentC}
	\ifthenelse{\isundefined{\studentD}}{}{\\ \studentD}
	\ifthenelse{\isundefined{\studentE}}{}{\\ \studentE}
    }
  \cfoot{}
  \rfoot{\small \textsc {Page \thepage\ of \pageref{LastPage}}}
  \renewcommand{\footrulewidth}{0.5pt}

\fancypagestyle{firststyle}
 {
  \fancyhf{}
   \renewcommand{\headrulewidth}{0pt}
   \chead{\includegraphics[width=5.5cm]{logoUU}}
   \rfoot{\small}
 }

\fancypagestyle{secondstyle}
 {
  \fancyhf{}
   \lhead{\includegraphics[width=4.5cm]{logoUU}}
   \rhead{\footnotesize \textsc {\\ \opdracht}}
   \lfoot
    {
	\footnotesize \studentA
	\ifthenelse{\isundefined{\studentB}}{}{\\ \studentB}
	\ifthenelse{\isundefined{\studentC}}{}{\\ \studentC}
	\ifthenelse{\isundefined{\studentD}}{}{\\ \studentD}
	\ifthenelse{\isundefined{\studentE}}{}{\\ \studentE}
    }
   \rfoot{\small}
 }

\setlength{\topmargin}{-0.3in}
\setlength{\textheight}{630pt}
\setlength{\headsep}{40pt}

%----------------------------------------------------------------------------------------
%	DOCUMENT INFORMATIE
%----------------------------------------------------------------------------------------
%Geef bij ieder command het juiste argument voor deze opdracht. Vul het hier in en het komt op meerdere plekken in het document correct te staan.

\newcommand{\titel}{Large Scale Fluid Simulation using Point Clouds}			%Zelfbedachte titel
\newcommand{\opdracht}{Thesis}		%Naam van opdracht die je van docent gehad hebt
\newcommand{\supervisor}{Dr. A. Vaxman}
\newcommand{\datum}{\today}					%Pas aan als je niet de datum van vanaag wilt hebben
\newcommand{\studentA}{Jack van der Drift}
\newcommand{\uunetidA}{4098978}

%----------------------------------------------------------------------------------------
%	AUTOMATISCHE TITEL
%----------------------------------------------------------------------------------------
\begin{document}
\thispagestyle{firststyle}
\begin{center}
	\textsc{\Large \opdracht}\\[0.2cm]
		\rule{\linewidth}{0.5pt} \\[0.4cm]
			{ \huge \bfseries \titel}
		\rule{\linewidth}{0.5pt} \\[0.2cm]
	{\large \datum  \\[0.4cm]}
	
	\begin{minipage}{0.4\textwidth}
		\begin{flushleft} 
			\emph{Student:}\\
			{\studentA \\ {\small \uunetidA \\[0.2cm]}}
		\end{flushleft}
	\end{minipage}
~
	\begin{minipage}{0.4\textwidth}
		\begin{flushright} 
			\emph{Supervisor:} \\
			\supervisor \\[0.2cm]
		\end{flushright}
	\end{minipage}\\[1 cm]
\end{center}

%----------------------------------------------------------------------------------------
%	INHOUDSOPGAVE EN ABSTRACT 
%----------------------------------------------------------------------------------------
\newpage
\thispagestyle{secondstyle}
\tableofcontents
\newpage
\thispagestyle{secondstyle}
\begin{abstract}
%\lorem[13]
\end{abstract}
\newpage

\pagestyle{fancy}
%----------------------------------------------------------------------------------------
%	INTRODUCTION
%----------------------------------------------------------------------------------------
\setcounter{page}{1}
\section{Introduction}

%\begin{itemize}
%\item Problem/challenge
%\item Why in general is this challenge not solved so far
%\item What does it contribute to the field
%\item How are we going to solve it (abstract way)
%\end{itemize}

%Challenge
Point clouds have become more and more popular to describe real life objects en scenes. (Find out exactly how they are being made and say something about it).

%Why aren't point clouds being used as of yet
Some of the reasons why point clouds aren't being used in fluid simulations yet are: 
\begin{itemize}
\item Point clouds from landscapes and coastlines usually contain several thousands to millions of points. Thus resulting in a lot of computation time, which makes them impractical for usage in fluid simulation compared to normal scenes consistent of mesh objects.
\item (more reasons, like availability, lack of research into use of point clouds)
\item (when they are getting used they use the reconstructed meshes. along the same line: Meshes are easier to use, as in more research and methods exist)
\end{itemize}

%Why it is beneficial to use them


%How we are going to solve it:
We present a novel way to use point clouds in flood simulation. In our simulation we use the original point cloud or one derived from it to compute a flood simulation. 

\subsection{Challenges}

\subsection{Research Question}

\subsection{Contributions}
Our contributions are: (finalized at the end for now a sort of goal for this thesis)
\begin{itemize}
\item An optimized way to scale down point clouds to a more usable size in flood simulations, while still maintaining good enough detail
\item Highly optimized flood simulation based upon the Lagrangian method which uses a point cloud for the scene.
\item a fast collision detection method between particles(points) and point clouds.
\end{itemize}

%----------------------------------------------------------------------------------------
%	LITERATURE REVIEW
%----------------------------------------------------------------------------------------

\section{Related Work}
(maybe needs more text as an introduction to the related work)

For our solution we need research from two different (fields): Research on fluid simulations and research on point clouds.

For the fluid simulation part we will be looking at what methods are available and which are applicable in our scope, meaning that they have to be able to handle large volumes and have the ability to be run in real-time. 

As for the second part, the point cloud part, we look at methods to scale down the amount of point and methods for fast collision detection, which both contribute to a real-time simulation.
%Literature review of state of the art papers

% Papers:
% 		- who do something similar
%		- who are competitors
%		- who provide background

\subsection{Fluid Simulation}
When it comes to fluids in simulations there are two main approaches: Eulerian and Lagrangian. (continue explanation)

\subsubsection{Particle Based}
(better explanation than above)
(Some recent methods (SPH etc))
(something about why or why not for our goals (so based on runtime))

\subsubsection{Grid Based}
(better explanation than above)
(Some recent methods (MAC grid etc))
(something about why or why not for our goals (so based on runtime))

\subsection{Point Clouds}
(maybe small introduction on point clouds)

\subsubsection{Collision Detection}
(Explain collision detection)
(Recent methods)
(explain what we can use from it)

\subsubsection{Scalability}
 (first explain the different types of scaling down the amount of particles, including papers which do it/explain the method)
\begin{itemize}
\item Octree
\item Range
\item Interpolation
\end{itemize}

(maybe explain something about parallelism and point clouds)

%----------------------------------------------------------------------------------------
%	BACKGROUND
%----------------------------------------------------------------------------------------

\section{Background}

%explain everything we that is needed

%----------------------------------------------------------------------------------------
%	METHOLOGY
%----------------------------------------------------------------------------------------

\section{Methology}

% How we are going to implement it

%----------------------------------------------------------------------------------------
%	RESULTS
%----------------------------------------------------------------------------------------

\section{Results}

% Eventual results which we will get (like runtime/frame rate etc)

%----------------------------------------------------------------------------------------
%	REFERENTIES
%----------------------------------------------------------------------------------------
%Meer informatie hierover volgt in blok 5 van jaar 1.

\nocite{*}
\bibliographystyle{acm}
%\bibliographystyle{apalike}
\bibliography{bib}

%----------------------------------------------------------------------------------------
%	BIJLAGEN
%----------------------------------------------------------------------------------------

%\section{Bijlage A}
%\section{Bijlage B}
%\section{Bijlage C}

\end{document}