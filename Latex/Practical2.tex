%----------------------------------------------------------------------------------------
%	PACKAGES EN DOCUMENT CONFIGURATIE
%----------------------------------------------------------------------------------------

\documentclass[a4paper,12pt]{article}
\usepackage{graphicx}
\usepackage[english]{babel}
\usepackage{fancyhdr}
\usepackage{lastpage}
\usepackage{xifthen}
\usepackage{algorithm2e}
\usepackage{lipsum}
\usepackage{hyperref}
\usepackage[utf8]{inputenc}
\usepackage{amsmath}
\usepackage{amssymb}
\usepackage{float}
\usepackage{multirow}
%----------------------------------------------------------------------------------------
%	HEADER & FOOTER
%----------------------------------------------------------------------------------------
\pagestyle{fancy}
  \lhead{\includegraphics[width=4.5cm]{logoUU}}		%Zorg dat het logo in dezelfde map staat
  \rhead{\footnotesize \textsc {\\ \opdracht}}
  \lfoot
    {
	\footnotesize \studentA
	\ifthenelse{\isundefined{\studentB}}{}{\\ \studentB}
	\ifthenelse{\isundefined{\studentC}}{}{\\ \studentC}
	\ifthenelse{\isundefined{\studentD}}{}{\\ \studentD}
	\ifthenelse{\isundefined{\studentE}}{}{\\ \studentE}
    }
  \cfoot{}
  \rfoot{\small \textsc {Page \thepage\ of \pageref{LastPage}}}
  \renewcommand{\footrulewidth}{0.5pt}

\fancypagestyle{firststyle}
 {
  \fancyhf{}
   \renewcommand{\headrulewidth}{0pt}
   \chead{\includegraphics[width=5.5cm]{logoUU}}
   \rfoot{\small \textsc {Page \thepage\ of \pageref{LastPage}}}
 }

\setlength{\topmargin}{-0.3in}
\setlength{\textheight}{630pt}
\setlength{\headsep}{40pt}

%----------------------------------------------------------------------------------------
%	DOCUMENT INFORMATIE
%----------------------------------------------------------------------------------------
%Geef bij ieder command het juiste argument voor deze opdracht. Vul het hier in en het komt op meerdere plekken in het document correct te staan.

\newcommand{\titel}{Large Scale Fluid Simulation using Point Clouds}			%Zelfbedachte titel
\newcommand{\opdracht}{Thesis}		%Naam van opdracht die je van docent gehad hebt
\newcommand{\supervisor}{Dr. A. Vaxman}
\newcommand{\datum}{\today}					%Pas aan als je niet de datum van vanaag wilt hebben
\newcommand{\studentA}{Jack van der Drift}
\newcommand{\uunetidA}{4098978}

%----------------------------------------------------------------------------------------
%	AUTOMATISCHE TITEL
%----------------------------------------------------------------------------------------
\begin{document}
\thispagestyle{firststyle}
\begin{center}
	\textsc{\Large \opdracht}\\[0.2cm]
		\rule{\linewidth}{0.5pt} \\[0.4cm]
			{ \huge \bfseries \titel}
		\rule{\linewidth}{0.5pt} \\[0.2cm]
	{\large \datum  \\[0.4cm]}
	
	\begin{minipage}{0.4\textwidth}
		\begin{flushleft} 
			\emph{Student:}\\
			{\studentA \\ {\small \uunetidA \\[0.2cm]}}
		\end{flushleft}
	\end{minipage}
~
	\begin{minipage}{0.4\textwidth}
		\begin{flushright} 
			\emph{Supervisor:} \\
			\supervisor \\[0.2cm]
		\end{flushright}
	\end{minipage}\\[1 cm]
\end{center}

%----------------------------------------------------------------------------------------
%	INHOUDSOPGAVE EN ABSTRACT 
%----------------------------------------------------------------------------------------
\newpage
\tableofcontents
\newpage
\begin{abstract}
%\lorem[13]
\end{abstract}
\newpage
%----------------------------------------------------------------------------------------
%	INTRODUCTION
%----------------------------------------------------------------------------------------

\section{Introduction}

\begin{itemize}
\item Problem/challenge
\item Why in general is this challenge not solved so far
\item What does it contribute to the field
\item How are we going to solve it (abstract way)
\end{itemize}
%----------------------------------------------------------------------------------------
%	LITERATURE REVIEW
%----------------------------------------------------------------------------------------

\section{Literature Review}

%Literature review of state of the art papers

% Papers:
% 		- who do something similar
%		- who are competitors
%		- who provide background

%----------------------------------------------------------------------------------------
%	BACKGROUND
%----------------------------------------------------------------------------------------

\section{Background}

%explain everything we that is needed

%----------------------------------------------------------------------------------------
%	METHOLOGY
%----------------------------------------------------------------------------------------

\section{Methology}

% How we are going to implement it

%----------------------------------------------------------------------------------------
%	RESULTS
%----------------------------------------------------------------------------------------

\section{Results}

% Eventual results which we will get (like runtime/frame rate etc)

%----------------------------------------------------------------------------------------
%	REFERENTIES
%----------------------------------------------------------------------------------------
%Meer informatie hierover volgt in blok 5 van jaar 1.

\nocite{*}
\bibliographystyle{acm}
%\bibliographystyle{apalike}
\bibliography{bib}

%----------------------------------------------------------------------------------------
%	BIJLAGEN
%----------------------------------------------------------------------------------------

%\section{Bijlage A}
%\section{Bijlage B}
%\section{Bijlage C}

\end{document}